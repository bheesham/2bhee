% E. Dunham -- Resume
% Contents Copyright (C) 2014 - 2015, E. Dunham

% LaTeX code for rendering the resume is distributed under the MIT license.
% See LICENSE.txt. It means you can use the code for whatever you want,
% including your own resume, but I'm not liable if it catches your computer on
% fire.

% Template originally developed by E. Dunham
% https://github.com/edunham/resume/blob/master/resume.tex

\documentclass[11pt]{article} % Set default font size to 11pt
\usepackage[normalem]{ulem} % for the underlines
\usepackage[compact]{titlesec} % Shrink default spacings
\usepackage{tabto} % For aligning skills section
\usepackage{hyperref}

\textwidth=7in
\textheight=10.5in
\topmargin -1.25in % Reclaim the default whitespace from top of page
\oddsidemargin -.25in % Reclaim whitespace on left, make it look centered
\pagenumbering{gobble} % Don't number pages
\setlength{\parindent}{0pt} % Don't indent paragraphs
\newcommand{\heading}[1]{
    %\section*{\centering\uline{\hfill #1 \hfill }} % Center the headings
    \section*{\uline{\hfill #1 }} % Right-align the headings
}
\newcommand{\squish}{
\setlength{\itemsep}{0.2pt}
    \setlength{\parskip}{0pt} % tweak parskip value to adjust total height
    \setlength{\parsep}{0.2pt}
}
\newcommand{\when}[1]{ % naming this 'date' would conflict with builtins
    \hfill \texttt{#1}
}
\newcommand{\experience}[3]{ % place, optional title, date
    \ifx&#2&
\item[{#1}]
    \else
\item[{#1}, \emph{#2}]
    \fi
    \when{#3}
}
\newcommand{\contact}[3]{
    \centerline{ \large \texttt{#1} $\bullet$ \texttt{#2} $\bullet$ \emph{#3} }
    \vspace{0.1in}
}
\newcommand{\skill}[2]{
    \textbf{#1} \tabto{2.5in} #2
}
\newcommand{\technologies}[1]{
    {\small \textbf{Technologies}: #1.}
}
% Write C++ all fancy-like
% http://www.parashift.com/c++-faq-lite/latex-macros.html
\newcommand{\CPP}{
    C\hspace{-.05em}\raisebox{.4ex}{\tiny\bf +}\hspace{-.10em}\raisebox{.4ex}{\tiny\bf +}
}

\begin{document}

\centerline{{\huge \bf Bheesham Persaud}}
\bigskip

\contact{me@bheesham.com}
        {(613) 612 - 5847}
        {Waterloo, Canada}

Full-stack developer looking to solve challenges related to managing large
scale networks.

\heading{Employment}%%%%%%%%%%%%%%%%%%%%%%%%%%%%%%%%%%%%%%%%%%%%%%%%%%%%%%%%%%

\begin{description}
        \squish

        \experience{Auvik Networks}
                   {Software Developer}
                   {10/2017 - Present}

        Stuff.

        \technologies{Scala, Python, Docker, SNMP, AWS (S3, EC2)}

        \experience{CENX}
                   {Software Developer}
                   {05/2016 - 09/2017}

        Developed service assurance solutions for telecommunication companies.

        Improved the deployment story for new components in our technology
        stack.

        Improved the development and testing workflow by emulating the
        production environment.

        \technologies{Clojure, ZooKeeper, Kafka, WildFly, Datomic, PostgreSQL,
        RethinkDB, Docker}

        \experience{Shared Services Canada}
                   {Developer / Analyst}
                   {04/2015 - 01/2016}

        Optimized programs that monitor the health and usage of large networks
        by reducing disk I/O and redundant steps.
        
        Created intranet websites which generate reports about network outages
        and changes.
        
        \technologies{Solaris 10, C\#, JavaScript, Perl 5, SNMP, SAS,
        Microsoft SQL Server}

        \experience{Carleton University}
                   {Research Assistant}
                   {05/2013 - 12/2014}

        Researched methods to increase software diversity by creating
        permutations of the OpenSSL library.

        Researched and implemented gesture based authentication for Android and
        iOS.
        
        \technologies{Java, Objective-C, JavaScript, Python, PHP, C/\CPP,
        \{Open, Boring, Libre\}SSL, \LaTeX}

        \experience{Carleton Computer Science Society}
                   {Director At Large: Services}
                   {05/2012 - 05/2014}

        Maintained a variety of services the society ran, such as website, DNS,
        and forum.

        \technologies{Ubuntu Server, NGINX, PowerDNS, Python, Plone}

        \experience{Canadian Association of Physicists}
                   {Webmaster}
                   {01/2012 - 04/2013}

        Maintained the Drupal website along with its many plugins and the
        Members only area.

        Fixed security flaws in legacy PHP and ASP systems.

        Created new services to manage the release of magazines on a quarterly
        basis.

        \technologies{ASP.Net, PHP, Kohana, CodeIgniter, jQuery, JavaScript,
        Drupal, Moneris}
\end{description}

\heading{Open-source Contributions}%%%%%%%%%%%%%%%%%%%%%%%%%%%%%%%%%%%%%%%%%%%%%%

\begin{description}
        \squish
        \experience{Redox}
                   {\url{https://www.redox-os.org}}
                   {}

            Fixed bugs in the assembly which allowed the operating system to
            run on the x86-64 architecture.

        \experience{xbps}
                   {\url{http://www.voidlinux.eu/}}
                   {}

            Improved documentation for xbps, the package manager of the Void
            (Linux) distribution.

        \experience{The Rust Programming Language}
                   {\url{http://rust-lang.org/}}
                   {}

            Improved the project's build scripts and documentation.

        \experience{The Rust Project's Buildbot}
                   {\url{http://buildbot.rust-lang.org/}}
                   {}

            Configured Buildbot to sign all release builds automatically and added
            basic authentication to the instance.
\end{description}

\heading{Academics}%%%%%%%%%%%%%%%%%%%%%%%%%%%%%%%%%%%%%%%%%%%%%%%%%%%%%%%%%%%%

\begin{description}
        \squish
        \experience{Carleton University}
                   {}
                   {09/2011 - 02/2017}

            Bachelor of Computer Science: Computer \& Internet Security Stream,
            Honours \\
            \textbf{Thesis}: FrankenSSL: Recombining Cryptographic Libraries to
            Create Software Variants \\
            \textbf{Supervisor}: Dr. Anil Somayaji

\end{description}

\heading{Publications}%%%%%%%%%%%%%%%%%%%%%%%%%%%%%%%%%%%%%%%%%%%%%%%%%%%%%%%%%%

\begin{description}
        \squish
        \experience{}
                   {}
                   {08/2016}

            B. Persaud, B. Obada-Obieh, N. Mansourzadeh, A. Moni, and A.
            Somayaji.  FrankenSSL: Recombining Cryptographic Libraries for
            Software Diversity.  \textit{Proceedings of the 11th Annual
            Symposium On Information Assurance}.
\end{description}

\end{document}
